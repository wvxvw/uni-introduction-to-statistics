% Created 2015-04-11 Sat 23:45
\documentclass[11pt]{article}
\usepackage[utf8]{inputenc}
\usepackage[T1]{fontenc}
\usepackage{fixltx2e}
\usepackage{graphicx}
\usepackage{longtable}
\usepackage{float}
\usepackage{wrapfig}
\usepackage{rotating}
\usepackage[normalem]{ulem}
\usepackage{amsmath}
\usepackage{textcomp}
\usepackage{marvosym}
\usepackage{wasysym}
\usepackage{amssymb}
\usepackage{hyperref}
\tolerance=1000
\usepackage[utf8]{inputenc}
\usepackage[usenames,dvipsnames]{color}
\usepackage[backend=bibtex, style=numeric]{biblatex}
\usepackage[scientific-notation=true]{siunitx}
\usepackage{commath}
\usepackage{mathtools}
\usepackage{marginnote}
\usepackage{listings}
\usepackage{color}
\usepackage{enumerate}
\hypersetup{urlcolor=blue}
\hypersetup{colorlinks,urlcolor=blue}
\addbibresource{bibliography.bib}
\setlength{\parskip}{16pt plus 2pt minus 2pt}
\definecolor{codebg}{rgb}{0.96,0.99,0.8}
\definecolor{codestr}{rgb}{0.46,0.09,0.2}
\author{Oleg Sivokon}
\date{\textit{<2015-03-27 Fri>}}
\title{Assignment 12, Introduction to Statistics}
\hypersetup{
  pdfkeywords={Probabilities, assignment},
  pdfsubject={Second asssignment in the course Introduction to Statistics},
  pdfcreator={Emacs 25.0.50.1 (Org mode 8.2.2)}}
\begin{document}

\maketitle
\tableofcontents


\lstset{ %
  backgroundcolor=\color{codebg},
  basicstyle=\ttfamily\scriptsize,
  breakatwhitespace=false,         % sets if automatic breaks should only happen at whitespace
  breaklines=false,
  captionpos=b,                    % sets the caption-position to bottom
  framexleftmargin=10pt,
  xleftmargin=10pt,
  framerule=0pt,
  frame=tb,                        % adds a frame around the code
  keepspaces=true,                 % keeps spaces in text, useful for keeping indentation of code (possibly needs columns=flexible)
  keywordstyle=\color{blue},       % keyword style
  showspaces=false,                % show spaces everywhere adding particular underscores; it overrides 'showstringspaces'
  showstringspaces=false,          % underline spaces within strings only
  showtabs=false,                  % show tabs within strings adding particular underscores
  stringstyle=\color{codestr},     % string literal style
  tabsize=2,                       % sets default tabsize to 2 spaces
}

\clearpage

\section{Problems}
\label{sec-1}

\subsection{Problem 1}
\label{sec-1-1}
Given the lottery ticket can have six numbers chosen from 1 through 6.
Each play selects a six digits number and the players are awarded according
to the number of digits they guessed.

\begin{enumerate}
\item What is the chance of guessing all numbers?
\item What is the chance of guessing exactly three of all numbers?
\item What is the chance of the winning number to be a palindrome?
\end{enumerate}

\subsubsection{Answer 1}
\label{sec-1-1-1}
The chance of guessing all numbers can be calculated as a product of
probabilities of guessing each number independently. Probability of
guessing one number is one in six, thus the total probability of
guessing the number is $\frac{1}{6^6} = \num{0.00002143347}$.
\subsubsection{Answer 2}
\label{sec-1-1-2}
The probability of guessing exactly three numbers is the probability
of guessing three numbers times the probability of guessing other three
not winning numbers, as many times as we can choose combinations of three
out of six, i.e.:
$\binom{6}{3} \times \frac{1}{6^3} \times \frac{5^3}{6^3} = \num{0.053583678}$.
\subsubsection{Answer 3}
\label{sec-1-1-3}
The probability of a six-digit number being a palindrome is the product
of first and last numbers being the same, second and fifth being the same
and third and fourth being the same.  Observe now that the condition of
being the same is equivalent to requiring that one of the numbers of the
pair be exactly of the six possible results, hence the probability of
two given numbers matching is exactly $\frac{1}{6}$, thus total probability
is $\frac{1}{6^3} = \num{0.0046296297}$.

Here's the calculation that verifies the results:

\lstset{language=Lisp,numbers=none}
\begin{lstlisting}
(defun generate-ticket ()
  (loop :repeat 6 :collect (random 6)))

(defun exactly-3-match (a b)
  (= 3 (loop :for i :in a :for j :in b
          :when (= i j) :count 1)))

(defun palindromep (tested) (equal tested (reverse tested)))

(defun num->ticket (n)
  (nreverse
   (loop :repeat 6 :collect (mod n 6) :do (setf n (floor n 6)))))

(defun ticket->num (ticket)
  (reduce (lambda (a b) (+ (* 6 a) b)) ticket :initial-value 0))

(defun next-ticket (previous)
  (num->ticket (1+ (ticket->num previous))))

(defparameter *all-tickets* (expt 6 6))

(defun chance-of-winning ()
  (/ (loop :with ticket := (generate-ticket)
        :repeat *all-tickets*
        :for attempt := '(0 0 0 0 0 0) :then (next-ticket attempt)
        :when (equal attempt ticket) :count 1)
     *all-tickets*))

(defun chance-of-three-matching ()
  (/ (loop :with ticket := (generate-ticket)
        :repeat *all-tickets*
        :for attempt := '(0 0 0 0 0 0) :then (next-ticket attempt)
        :when (exactly-3-match ticket attempt) :count 1)
     *all-tickets*))

(defun chance-of-palindrome ()
  (/ (loop :repeat *all-tickets*
        :for attempt := '(0 0 0 0 0 0) :then (next-ticket attempt)
        :when (palindromep attempt) :count 1)
     *all-tickets*))

(format t "~&Chance of winning the lotery:     ~f~%~
             Chance of guessing exactly three: ~f~%~
             Chance of palindrome ticket:      ~f"
        (chance-of-winning)
        (chance-of-three-matching)
        (chance-of-palindrome))
\end{lstlisting}

\begin{verbatim}
Chance of winning the lotery:     0.00002143347
Chance of guessing exactly three: 0.053583678
Chance of palindrome ticket:      0.0046296297
\end{verbatim}
\subsection{Problem 2}
\label{sec-1-2}
Given five country flags, four town flags and two army flags all hung
together on a thread.

\begin{enumerate}
\item What is the chance that three first flags are the country flags?
\item What is the chance that all the flags of the same kind hung together?
\item What is the chance that between two army flags, there will be only
the country flags?
\item If three flags selected at random, what is the chance that at least
two of them are of the same kind?
\end{enumerate}

\subsubsection{Answser 4}
\label{sec-1-2-1}
The chance of the first three flags being the country flags is the chance
of the first being the country flag, times second, times third.
The chance of the first is 5 in 11, the chance of second is 4 in 10 and
the chance of third is 3 in 9.  Thus the total chance is
$\frac{5 * 4 * 3}{11 * 10 * 9} = \num{0.0606060606058}$.
\subsubsection{Answer 5}
\label{sec-1-2-2}
Observe, first, that there are only 3! possibilities for such arrangments,
i.e.
\begin{enumerate}
\item \textbf{country}, \textbf{city}, \textbf{army}.
\item \textbf{country}, \textbf{army}, \textbf{city}.
\item \ldots{}
\item \textbf{army}, \textbf{city}, \textbf{country}.
\end{enumerate}

Given the total number of ways the flags can be hung:
\begin{equation*}
  \frac{11!}{5!4!2!} = 6930,
\end{equation*}

it gives that there is only $\frac{3!}{6930} = \num{0.000865800865801}$
chance the flags will hang in the specified order.
\subsubsection{Answer 6}
\label{sec-1-2-3}
The number of ways country flags can be hung between the army flags are
either one, or two, or three, or four, or five, for each case there is
a number of ways the group of flags can be positioned on the rope.
\begin{enumerate}
\item There are $11-3+1=9$ ways to position the flags on the rope.
\item There are $11-4+1=8$ ways to position the flags on the rope.
\item There are $11-5+1=7$ ways to position the flags on the rope.
\item There are $11-6+1=6$ ways to position the flags on the rope.
\item There are $11-7+1=5$ ways to position the flags on the rope.
\end{enumerate}

Also observe that for each case, the remaining combinations of flags
are defined by the number of flags we can permute times the number of
ways we can position the army flags.

\begin{enumerate}
\item $\frac{8!}{4!4!} = 70$.
\item $\frac{7!}{3!4!} = 35$.
\item $\frac{6!}{2!4!} = 15$.
\item $\frac{5!}{1!4!} = 5$.
\item $\frac{4!}{4!} = 1$.
\end{enumerate}

Summing this all up gives:
$\frac{70*9+35*8+15*7+5*6+1*5}{6930} = \num{0.151515151515}$.
\subsubsection{Answer 7}
\label{sec-1-2-4}
We can divide this problem into two sub-problems:
\begin{enumerate}
\item In how many ways can we slect three first flags s.t. the first
ant the second or first and the last will match.  We will have
thee disjoint probabilities for each kind of flag weighted by
their relative probablity:
\begin{equation*}
  \begin{aligned}
    \frac{5}{11} \times \Big(\frac{4}{10} + \frac{4}{9}\Big)
    \times{4 + 2}{10}  &= \num{0.30303030303} \\
    \frac{4}{11} \times \Big(\frac{3}{10} + \frac{3}{9}\Big)
    \times {5 + 2}{10} &= \num{0.193939393939} \\
    \frac{2}{11} \times \Big(\frac{1}{10} + \frac{1}{9}\Big)
    \times {5 + 4}{10} &= \num{0.0363636363636}.
  \end{aligned}
\end{equation*}

\item And the probability that the last two flags are the same, this
probability is again weighted by the first flag selected and
summed for two remaining kinds of flags:
\begin{equation*}
  \begin{aligned}
    \frac{5}{11} \times \Big(\frac{4}{10} \times \frac{3}{9} +
    \frac{2}{10} \times \frac{1}{9}\Big) &= \num{0.0707070707067} \\
    \frac{4}{11} \times \Big(\frac{5}{10} \times \frac{4}{9} +
    \frac{2}{10} \times \frac{1}{9}\Big) &= \num{0.0888888888886} \\
    \frac{2}{11} \times \Big(\frac{5}{10} \times \frac{4}{9} +
    \frac{4}{10} \times \frac{3}{9}\Big) &= \num{0.0646464646463}.
  \end{aligned}
\end{equation*}
\end{enumerate}

Summing it up gives 
\begin{equation*}
  \begin{aligned}
    \num{0.0707070707067} &+ \\
    \num{0.0888888888886} &+ \\
    \num{0.0646464646463} &+ \\
    \num{0.30303030303}   &+ \\
    \num{0.193939393939}  &+ \\
    \num{0.0363636363636} &= \num{0.757575757575}.
  \end{aligned}
\end{equation*}

The code to verify the answers:
\lstset{language=Lisp,numbers=none}
\begin{lstlisting}
(defun shift-elements (vec low high)
  (prog1 vec
    (loop :for i :from high :downto low :do
       (setf (aref vec (1+ i)) (aref vec i)))))

(defun initialize-perms (vec element &optional (low 0))
  (prog1 vec
    (loop :with j := low
       :for i :from low :below (length vec)
       :if (eql (aref vec i) element) :do
       (shift-elements vec j (1- i))
       (setf (aref vec j) element j (1+ j)))))

(defun can-move-index (vec element)
  (loop :for i :from (1- (length vec)) :downto 0
     :for current := (aref vec i)
     :with prev := nil
     :when (and prev
                (not (eql prev element))
                (eql current element))
     :do (return i)
     :end :do (setf prev current)))

(defun move-index (vec index)
  (prog1 vec
    (psetf (aref vec (1+ index)) (aref vec index)
           (aref vec index) (aref vec (1+ index)))))

(defun permute-group (vec element &optional (low 0))
  (cons
   (copy-seq (initialize-perms vec element low))
   (loop :with init := (initialize-perms vec element low)
      :with last := low
      :for moving := (can-move-index init element)
      :while moving
      :do (move-index init moving)
      :when (< moving last) :do
      (initialize-perms init element (1+ moving))
      :end
      :collect (copy-seq init)
      :do (setf last moving))))

(defun canonical (element repeat &optional (previous #()))
  (loop :with result := (make-array (+ repeat (length previous)))
     :for i :below repeat :do
     (setf (aref result i) element)
     :finally 
     (return
       (prog1 result
         (loop :for j :from i :below (length result) :do
            (setf (aref result j) (aref previous (- j i))))))))

(defun permutations-with-repetition (groups)
  (loop :with first := (car groups)
     :with perms := (list (canonical (car first) (cdr first)))
     :for (key . value) :in (cdr groups)
     :do (setf perms
               (loop :for perm :in perms
                  :nconc (permute-group
                          (canonical key value perm) key)))
     :finally (return perms)))

(defparameter *all-flags*
  (permutations-with-repetition '((a . 5) (b . 4) (c . 2))))

(defun first-three-a ()
  (/ (loop :for flags :in *all-flags*
        :when (equal (coerce (subseq flags 0 3) 'list) '(a a a))
        :count 1)
     (length *all-flags*)))

(defun togetherp (flags)
  (= 2 (loop :with previous := nil
          :for elt :across flags
          :when (and previous (not (eql elt previous)))
          :count 1 :end
          :do (setf previous elt))))

(defun flags-hang-together ()
  (/ (loop :for flags :in *all-flags*
        :when (togetherp flags)
        :count 1)
     (length *all-flags*)))

(defun between-army-p (flags)
  (not
   (loop :with flags-seen := 0
      :with previous := nil
      :for elt :across flags :do
      (case elt
        (b (when (= flags-seen 1) (return t)))
        (c (when (eql previous 'c) (return t))
           (incf flags-seen)))
      (setf previous elt))))

(defun between-army ()
  (/ (loop :for flags :in *all-flags*
        :when (between-army-p flags)
        :count 1)
     (length *all-flags*)))

(defun first-three-duplicate ()
  (/ (loop :for flags :in *all-flags*
        :when (or (eql (aref flags 0) (aref flags 1))
                  (eql (aref flags 1) (aref flags 2))
                  (eql (aref flags 0) (aref flags 2)))
        :count 1)
     (length *all-flags*)))

(format t "~&Chance three first flags are country flags: ~f~%~
             Chance all flags hang together:             ~f~%~
             Chance only city flags between army flags:  ~f~%~
             Chance of duplicate in first three flags:   ~f~%"
        (first-three-a)
        (flags-hang-together)
        (between-army)
        (first-three-duplicate))
\end{lstlisting}

\begin{verbatim}
Chance three first flags are country flags: 0.060606062
Chance all flags hang together:             0.0008658009
Chance only city flags between army flags:  0.15151516
Chance of duplicate in first three flags:   0.75757575
\end{verbatim}
\subsection{Problem 3}
\label{sec-1-3}
Given five different research subject and ten students selecting from these
subjects at random:
\begin{enumerate}
\item What is the chance all students will select the same subject?
\item What is the chance there will be exactly 3 papers on the first subject?
\item What is the chance two students will select the same subject?
\item What is the chance that two particular students will choose the first subject,
thee students will choose the second subject, four will choose the third
subject and the rest will choose the fourth subject?
\item What is the chance that only two of the five subjects will be selected?
\end{enumerate}

\subsubsection{Answer 8}
\label{sec-1-3-1}
The probability of selecting the subject is independent for each student,
thus the total probability is the product of probabilities of each student
selecting the same subject times number of subjects.  This gives:
\begin{equation*}
  5 \times \frac{1}{5^{10}} = \num{0.000000512}.
\end{equation*}
\subsubsection{Answer 9}
\label{sec-1-3-2}
The chance of having exactly three papers written on the first subject is
the chance of writing thee papers times the chance of writing other papers
times the number of ways the students can be assigned to papers.  This
gives:
$\binom{10}{3} \times \frac{1}{5^3} \times \frac{4^7}{5^7} = \num{0.201326592}$.
\subsubsection{Answer 10}
\label{sec-1-3-3}
The chance of two students selecting a particular subject is the number of
awailable subjects times the probability of selecting the same subject for
two students.  I.e. $5 \times \frac{1}{5^2} = 0.2$.
\subsubsection{Answer 11}
\label{sec-1-3-4}


\subsubsection{Answer 12}
\label{sec-1-3-5}
The chance that only two of the five subjects will be selected is the chance
that every subsequent student selects from two of the possible subjects
(independently) times the number of ways the subjects can be paired.  This
gives $\Big(\frac{2}{5}\Big)^{10} \times 45 = \num{0.004718592}$.
% Emacs 25.0.50.1 (Org mode 8.2.2)
\end{document}