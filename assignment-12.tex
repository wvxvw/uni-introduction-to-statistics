% Created 2015-04-10 Fri 22:43
\documentclass[11pt]{article}
\usepackage[utf8]{inputenc}
\usepackage[T1]{fontenc}
\usepackage{fixltx2e}
\usepackage{graphicx}
\usepackage{longtable}
\usepackage{float}
\usepackage{wrapfig}
\usepackage{rotating}
\usepackage[normalem]{ulem}
\usepackage{amsmath}
\usepackage{textcomp}
\usepackage{marvosym}
\usepackage{wasysym}
\usepackage{amssymb}
\usepackage{hyperref}
\tolerance=1000
\usepackage[utf8]{inputenc}
\usepackage[usenames,dvipsnames]{color}
\usepackage[backend=bibtex, style=numeric]{biblatex}
\usepackage{commath}
\usepackage{mathtools}
\usepackage{marginnote}
\usepackage{listings}
\usepackage{color}
\usepackage{enumerate}
\hypersetup{urlcolor=blue}
\hypersetup{colorlinks,urlcolor=blue}
\addbibresource{bibliography.bib}
\setlength{\parskip}{16pt plus 2pt minus 2pt}
\definecolor{codebg}{rgb}{0.96,0.99,0.8}
\definecolor{codestr}{rgb}{0.46,0.09,0.2}
\author{Oleg Sivokon}
\date{\textit{<2015-03-27 Fri>}}
\title{Assignment 12, Introduction to Statistics}
\hypersetup{
  pdfkeywords={Probabilities, assignment},
  pdfsubject={Second asssignment in the course Introduction to Statistics},
  pdfcreator={Emacs 25.0.50.1 (Org mode 8.2.2)}}
\begin{document}

\maketitle
\tableofcontents


\lstset{ %
  backgroundcolor=\color{codebg},
  basicstyle=\ttfamily\scriptsize,
  breakatwhitespace=false,         % sets if automatic breaks should only happen at whitespace
  breaklines=false,
  captionpos=b,                    % sets the caption-position to bottom
  framexleftmargin=10pt,
  xleftmargin=10pt,
  framerule=0pt,
  frame=tb,                        % adds a frame around the code
  keepspaces=true,                 % keeps spaces in text, useful for keeping indentation of code (possibly needs columns=flexible)
  keywordstyle=\color{blue},       % keyword style
  showspaces=false,                % show spaces everywhere adding particular underscores; it overrides 'showstringspaces'
  showstringspaces=false,          % underline spaces within strings only
  showtabs=false,                  % show tabs within strings adding particular underscores
  stringstyle=\color{codestr},     % string literal style
  tabsize=2,                       % sets default tabsize to 2 spaces
}

\clearpage

\section{Problems}
\label{sec-1}

\subsection{Problem 1}
\label{sec-1-1}
Given the lottery ticket can have six numbers chosen from 1 through 6.
Each play selects a six digits number and the players are awarded according
to the number of digits they guessed.

\begin{enumerate}
\item What is the chance of guessing all numbers?
\item What is the chance of guessing exactly three of all numbers?
\item What is the chance of the winning number to be a palindrome?
\end{enumerate}

\subsubsection{Answer 1}
\label{sec-1-1-1}
The chance of guessing all numbers can be calculated as a product of
probabilities of guessing each number independently. Probability of
guessing one number is one in six, thus the total probability of
guessing the number is $\frac{1}{6^6} = 0.00002143347$.
\subsubsection{Answer 2}
\label{sec-1-1-2}
The probability of guessing exactly three numbers is the probability
of guessing three numbers times the probability of guessing other three
not winning numbers, as many times as we can choose combinations of three
out of six, i.e.:
$\binom{6}{3} \times \frac{1}{6^3} \times \frac{5^3}{6^3} = 0.053583678$.
\subsubsection{Answer 3}
\label{sec-1-1-3}
The probability of a six-digit number being a palindrome is the product
of first and last numbers being the same, second and fifth being the same
and third and fourth being the same.  Observe now that the condition of
being the same is equivalent to requiring that one of the numbers of the
pair be exactly of the six possible results, hence the probability of
two given numbers matching is exactly $\frac{1}{6}$, thus total probability
is $\frac{1}{6^3} = 0.0046296297$.

Here's the calculation that verifies the results:

\lstset{language=Lisp,numbers=none}
\begin{lstlisting}
(defun generate-ticket ()
  (loop :repeat 6 :collect (random 6)))

(defun exactly-3-match (a b)
  (= 3 (loop :for i :in a :for j :in b
          :when (= i j) :count 1)))

(defun palindromep (tested) (equal tested (reverse tested)))

(defun num->ticket (n)
  (nreverse
   (loop :repeat 6 :collect (mod n 6) :do (setf n (floor n 6)))))

(defun ticket->num (ticket)
  (reduce (lambda (a b) (+ (* 6 a) b)) ticket :initial-value 0))

(defun next-ticket (previous)
  (num->ticket (1+ (ticket->num previous))))

(defparameter *all-tickets* (expt 6 6))

(defun chance-of-winning ()
  (/ (loop :with ticket := (generate-ticket)
        :repeat *all-tickets*
        :for attempt := '(0 0 0 0 0 0) :then (next-ticket attempt)
        :when (equal attempt ticket) :count 1)
     *all-tickets*))

(defun chance-of-three-matching ()
  (/ (loop :with ticket := (generate-ticket)
        :repeat *all-tickets*
        :for attempt := '(0 0 0 0 0 0) :then (next-ticket attempt)
        :when (exactly-3-match ticket attempt) :count 1)
     *all-tickets*))

(defun chance-of-palindrome ()
  (/ (loop :repeat *all-tickets*
        :for attempt := '(0 0 0 0 0 0) :then (next-ticket attempt)
        :when (palindromep attempt) :count 1)
     *all-tickets*))

(format t "~&Chance of winning the lotery:     ~f~%~
             Chance of guessing exactly three: ~f~%~
             Chance of palindrome ticket:      ~f"
        (chance-of-winning)
        (chance-of-three-matching)
        (chance-of-palindrome))
\end{lstlisting}

\begin{verbatim}
Chance of winning the lotery:     0.00002143347
Chance of guessing exactly three: 0.053583678
Chance of palindrome ticket:      0.0046296297
\end{verbatim}
% Emacs 25.0.50.1 (Org mode 8.2.2)
\end{document}