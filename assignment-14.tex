% Created 2015-05-28 Thu 15:58
\documentclass[11pt]{article}
\usepackage[utf8]{inputenc}
\usepackage[T1]{fontenc}
\usepackage{fixltx2e}
\usepackage{graphicx}
\usepackage{longtable}
\usepackage{float}
\usepackage{wrapfig}
\usepackage{rotating}
\usepackage[normalem]{ulem}
\usepackage{amsmath}
\usepackage{textcomp}
\usepackage{marvosym}
\usepackage{wasysym}
\usepackage{amssymb}
\usepackage{capt-of}
\usepackage{hyperref}
\tolerance=1000
\usepackage[utf8]{inputenc}
\usepackage[usenames,dvipsnames]{color}
\usepackage[scientific-notation=true]{siunitx}
\usepackage{commath}
\usepackage{mathtools}
\usepackage{marginnote}
\usepackage{listings}
\usepackage{color}
\usepackage{enumerate}
\usepackage{systeme}
\hypersetup{urlcolor=blue}
\hypersetup{colorlinks,urlcolor=blue}
\setlength{\parskip}{16pt plus 2pt minus 2pt}
\definecolor{codebg}{rgb}{0.96,0.99,0.8}
\definecolor{codestr}{rgb}{0.46,0.09,0.2}
\author{Oleg Sivokon}
\date{\textit{<2015-05-24 Sun>}}
\title{Assignment 14, Introduction to Statistics}
\hypersetup{
 pdfauthor={Oleg Sivokon},
 pdftitle={Assignment 14, Introduction to Statistics},
 pdfkeywords={Conditional probabilities, assignment},
 pdfsubject={Third assignment in the course Introduction to Statistics},
 pdfcreator={Emacs 25.0.50.1 (Org mode 8.3beta)}, 
 pdflang={English}}
\begin{document}

\maketitle
\tableofcontents

  \lstset{ %
    backgroundcolor=\color{codebg},
    basicstyle=\ttfamily\scriptsize,
    breakatwhitespace=false,         % sets if automatic breaks should only happen at whitespace
    breaklines=false,
    captionpos=b,                    % sets the caption-position to bottom
    framexleftmargin=10pt,
    xleftmargin=10pt,
    framerule=0pt,
    frame=tb,                        % adds a frame around the code
    keepspaces=true,                 % keeps spaces in text, useful for keeping indentation of code (possibly needs columns=flexible)
    keywordstyle=\color{blue},       % keyword style
    showspaces=false,                % show spaces everywhere adding particular underscores; it overrides 'showstringspaces'
    showstringspaces=false,          % underline spaces within strings only
    showtabs=false,                  % show tabs within strings adding particular underscores
    stringstyle=\color{codestr},     % string literal style
    tabsize=2,                       % sets default tabsize to 2 spaces
  }

\clearpage

\section{Problems}
\label{sec:orgheadline24}

\subsection{Problem 1}
\label{sec:orgheadline5}
An ant travels 4 meters to the left with a chance 0.4 and 3 meters to the right
with a chance 0.6 every day.
\begin{enumerate}
\item What is the expected value and the variance of ant's position to the right
of starting point?
\item What is the probability of ant moving to the right for the first time
on the fourth day?
\item What is the chance the ant will walk 9 meters to the right by the end of
day ten?
\item What is the expected value and variance after ten days?
\end{enumerate}

\subsubsection{Answer 1}
\label{sec:orgheadline1}
\(E(x)=0.4 \cdot (-4) + 0.6 \cdot 3 = 1.8 - 1.6 = 0.2\).  In other words: the
expected value is just a weighted sum of all possible outcomes.

\subsubsection{Answer 2}
\label{sec:orgheadline2}
In order to move right for the first time on the fourth day the ant needs
to move left on three previous days, hence, using the product law:
\(P(x) = 0.4^3 = 0.064\).

\subsubsection{Answer 3}
\label{sec:orgheadline3}
The chance of moving 9 meters to the right is the chance of moving \(x\)
days to the left and \(y\) days to the right s.t. \(x \cdot (-4) + y \cdot 3 = 9\).
Since we also know that the ant needs to spend exactly 10 days to get
to its destination, we obtain a system of linear equations:
\begin{eqnarray*}
  \systeme*{-4x + 3y = 9, x + y = 10} \\
  x              &=& 10 - y \\
  3y - 4(10 - y) &=& 9 \\
  3y - 40 + 4y   &=& 9 \\
  7y             &=& 49 \\
  y              &=& 7 \\
  x              &=& 3\;.
\end{eqnarray*}

From this, the chance of getting 9 meters to the right by the end of
the tenth day is \(P(x) = 0.6^7 \cdot 0.4^3 = \num{0.0017915904}\).

\subsubsection{Answer 4}
\label{sec:orgheadline4}
The expected value by the day ten is just the sum of ten expected values of
the movement carried out in one day, thus \(E(10x) = 10E(x) = 10 \cdot 0.2 =
    2\).

\subsection{Problem 2}
\label{sec:orgheadline10}
Yarden rolls a four-sided fair die until he rolls 1.
\begin{enumerate}
\item What is the chance that he will roll the die exactly four times?
\item What is the expected value of times he will roll the die?
\end{enumerate}

Sharon, too, rolls the same die, but she will cease after rolling it just
six times.  What is the chance Sharon will not roll 1?

Alon rolls the same die.  He wants to roll 1 three times, not necessarily
in succession.  What is the chance he will roll the dice eight times?

\subsubsection{Answer 5}
\label{sec:orgheadline6}
It's either Yarden rolls 1 or he rolls 2, 3, or 4, since the die is fair,
the chance of rolling 1 is \(\frac{1}{4}\), the chance of not rolling 1 is
thus \(\frac{3}{4}\).  The chance of not rolling 1 for the first three times
and rolling it afterwards is \(\left(\frac{3}{4}\right)^3 \cdot \frac{1}{4} =
    \frac{27}{64 \cdot 4} = \frac{27}{256}\).

\subsubsection{Answer 6}
\label{sec:orgheadline7}
Since Yarden expects to roll 1 with a chance \(\frac{1}{4}\), he expects to
finish after \(\frac{x}{4}=1\) tries, i.e. on the fourth roll.

\subsubsection{Answer 7}
\label{sec:orgheadline8}
As discussed in \ref{sec:orgheadline6}, the chance of not rolling 1 is \(\frac{3}{4}\),
hence the chance of not rolling 1 even after eight trials is
\(\left(\frac{3}{4}\right)^8 = \frac{6561}{65536}\).

\subsubsection{Answer 8}
\label{sec:orgheadline9}
This is simply the chance of rolling 1 three times and rolling ``not 1''
five times, hence \(\left(\frac{3}{4}\right)^5 \cdot
    \left(\frac{1}{4}\right)^3 = \frac{234}{1024 \cdot 64} = \frac{243}{65536}\).

\subsection{Problem 3}
\label{sec:orgheadline14}
Yoel throws a ball at ten bottles standing next to each other.  With
probability of 0.1 he can scatter all of the bottles.  He will miss entirely
with a probability 0.2.  The bottles are rearranged after each throw.
\begin{enumerate}
\item What is the chance Yoel will hit all the bottles in exactly 2 throws?
\item What is the chance that in two throws Yoel will hit all the bottles,
in three throws he will miss entirely, and the rest he will neither
miss, nor hit all the bottles?
\item What is the expected value and the variance of the number of time Yoel
hitting all the bottles?
\end{enumerate}

\subsubsection{Answer 9}
\label{sec:orgheadline11}
The chance of hitting all the bottles in two tries is just the sum of
the chance of hitting all bottles on the first and on the second tries.
This is so because the events are independent.  Thus the total probability
is \(2 \cdot 0.1 = 0.2\).

\subsubsection{Answer 10}
\label{sec:orgheadline12}
Using the polynomial distribution, the chance of hitting all the bottles
twice, missing three times and neither missing nor hitting all the bottles
is \(\frac{10!}{2!3!5!} \cdot 0.2^3 \cdot (1 - 0.1 - 0.2)^{10 - 2 - 3} = 0.01
    \cdot 0.008 \cdot 0.16807 = \num{0.33882912}\).

\subsubsection{Answer 11}
\label{sec:orgheadline13}
Using binomial distribution formula: \(E(x)=pn\), expected value is \(10 \cdot
    0.1 = 1\).  The variance is \(V(x)= np(1 - p) = 1 \cdot 0.9 = 0.9\).

\subsection{Problem 4}
\label{sec:orgheadline19}
Yaron receives SMSs with expected value given by Poisson distribution of
5 messages per hour.
\begin{enumerate}
\item What is the chance that Yaron will receive 8 messages in two hours?
\item Yaron received 8 messages in two hours, what is the chance he received
only one message in the first hour?
\item Yaron defines an hour as ``bad'' if during that hour he received at most
one message.  He randomly selects hours and verifies the record of number
of messages received during that hour until he encounter the ``bad'' hour
for the first time.
\begin{itemize}
\item What is the chance he will have to look into four records?
\item What is the expected value of ``not bad'' hours?
\end{itemize}
\end{enumerate}

\subsubsection{Answer 12}
\label{sec:orgheadline15}
Since this is Poisson distribution we assume message arrivals to be uniformly
distributed along the time-line, hence \(\lambda = 5 \cdot 2 = 10\).  Thus, the
cance is calculated using \(p(X = k) = e^{-\lambda} \cdot \frac{\lambda^k}{k!}\).
Hence \(p(X) = e^{-8} \cdot \frac{10^8}{8!} = \num{0.832000565234}\).

\subsubsection{Answer 13}
\label{sec:orgheadline16}
Similar to the question above, new \(\lambda = 8 / 2 = 4\), the probability thus
is \(p(X) = e^{-4} \cdot \frac{4^1}{1!} = \num{0.0732625555548}\).

\subsubsection{Answer 14}
\label{sec:orgheadline17}
Since the expected value for Poisson distribution is equal to \(\lambda\) (5
in our case), Yaron should expect to receive one message per hour with the
chance \(p(X) = e^{-5} \cdot \frac{5^1}{1!} = \num{0.0336897349954}\).  Since
he looked in four records, the chance will be four times that high, i.e.
\(\num{0.134758939982}\).

\subsubsection{Answer 15}
\label{sec:orgheadline18}
In the previous answer we've calculated the expected value for ``bad'' hour,
its complement is the expectation for the ``not bad'' hour,
viz. \(\num{0.966310265005}\).

\subsection{Problem 5}
\label{sec:orgheadline23}
Omri has a 0.5 chance of having a meeting in development department.  He also
has a 0.2 chance of managerial meeting.  While at the same time, there will
be no meeting at all on that day with a chance of 0.4.  Whether Omri has a
managerial meeting is independent of whether he has to meet the developers.
\begin{enumerate}
\item Let \(X\) be the number of meetings Omri has to attend during the day,
find probability function \(p(X)\).
\item What is the chance that in 2 days out of 5 Omri will have to attend no
meeting whatsoever?
\item Suppose Omri had to attend at least one meeting on each day of the week,
what is the chance Omri had to attend two meetings on two days during
the same week?
\end{enumerate}

\subsubsection{Answer 16}
\label{sec:orgheadline20}
It is expedient that we first find the probability of both meetings happening
on the same day.  Let the even of having a managerial meeting be \(M\), and
the even of having a development meeting be \(D\).  The even of having no
meeting whatsoever will be denoted by \(N\), then
\(p(M \cap D) = p(N) + p(M) + p(D) - p(\Omega) = 0.4 + 0.2 + 0.5 - 1 = 0.1\).
Since all probabilities must sum up to one, the probability of having exactly
one meeting is \(1 - 0.4 - 0.1\).  Hence, the probability function is given by:

\begin{center}
\begin{tabular}{lrrr}
\(x\) & 0 & 1 & 2\\
\hline
\(p(X)\) & 0.4 & 0.5 & 0.1\\
\end{tabular}
\end{center}

\subsubsection{Answer 17}
\label{sec:orgheadline21}
The answer can be given using binomial distribution formula:
\(X \sim B(n, p)\), where \(n\) is the number of trials (5), and \(k\) is the number
of successes (2):
\begin{eqnarray*}
  {n \choose k} p^k (1 - p)^{n - k} &=& {5 \choose 2} 0.4^2 \cdot 0.6^3 \\
                                   &=& 10 \cdot 0.16 \cdot 0.216 \\
                                   &=& 0.3456\;.
\end{eqnarray*}

\subsubsection{Answer 18}
\label{sec:orgheadline22}
A way to look at this problem could be as follows: a chance of having some
meetings (either one or two) is \(p(M \cup D) = 0.6\).  The chance of having
two meetings is \(p(M \cap D) = 0.1\) (as we already calculated in \ref{sec:orgheadline20}.
That is a chance of having two meetings, provided we know some meeting took
place is one in six.  If we try it five times, then our chances grow
five-fold, viz. \(\frac{1}{6} \cdot 5 = \frac{5}{6}\) for one meeting, and
half of that for two meetings: \(\frac{5}{12}\).
\end{document}