% Created 2015-05-03 Sun 13:14
\documentclass[11pt]{article}
\usepackage[utf8]{inputenc}
\usepackage[T1]{fontenc}
\usepackage{fixltx2e}
\usepackage{graphicx}
\usepackage{longtable}
\usepackage{float}
\usepackage{wrapfig}
\usepackage{rotating}
\usepackage[normalem]{ulem}
\usepackage{amsmath}
\usepackage{textcomp}
\usepackage{marvosym}
\usepackage{wasysym}
\usepackage{amssymb}
\usepackage{hyperref}
\tolerance=1000
\usepackage[utf8]{inputenc}
\usepackage[usenames,dvipsnames]{color}
\usepackage[backend=bibtex, style=numeric]{biblatex}
\usepackage[scientific-notation=true]{siunitx}
\usepackage{commath}
\usepackage{mathtools}
\usepackage{marginnote}
\usepackage{listings}
\usepackage{color}
\usepackage{enumerate}
\hypersetup{urlcolor=blue}
\hypersetup{colorlinks,urlcolor=blue}
\addbibresource{bibliography.bib}
\setlength{\parskip}{16pt plus 2pt minus 2pt}
\definecolor{codebg}{rgb}{0.96,0.99,0.8}
\definecolor{codestr}{rgb}{0.46,0.09,0.2}
\author{Oleg Sivokon}
\date{\textit{<2015-03-27 Fri>}}
\title{Assignment 13, Introduction to Statistics}
\hypersetup{
  pdfkeywords={Conditional probabilities, assignment},
  pdfsubject={Third asssignment in the course Introduction to Statistics},
  pdfcreator={Emacs 25.0.50.1 (Org mode 8.2.2)}}
\begin{document}

\maketitle
\tableofcontents


  \lstset{ %
    backgroundcolor=\color{codebg},
    basicstyle=\ttfamily\scriptsize,
    breakatwhitespace=false,         % sets if automatic breaks should only happen at whitespace
    breaklines=false,
    captionpos=b,                    % sets the caption-position to bottom
    framexleftmargin=10pt,
    xleftmargin=10pt,
    framerule=0pt,
    frame=tb,                        % adds a frame around the code
    keepspaces=true,                 % keeps spaces in text, useful for keeping indentation of code (possibly needs columns=flexible)
    keywordstyle=\color{blue},       % keyword style
    showspaces=false,                % show spaces everywhere adding particular underscores; it overrides 'showstringspaces'
    showstringspaces=false,          % underline spaces within strings only
    showtabs=false,                  % show tabs within strings adding particular underscores
    stringstyle=\color{codestr},     % string literal style
    tabsize=2,                       % sets default tabsize to 2 spaces
  }

\clearpage

\section{Problems}
\label{sec-1}

\subsection{Problem 1}
\label{sec-1-1}
Given the probabilities of two basketball teams $A$ and $B$ of
\begin{description}
\item[{winning}] $P(W | A) = 0.6$, $P(W | B) = 0.3$.
\item[{drawing}] $P(D) = 0.1$.
\end{description}


\begin{enumerate}
\item What is the chance the teams will play only two games to establish
the winner?
\item The third game was a draw.  What is the chance $A$ wins in the
first round?
\item Provided $A$ wins the tournament, what is the chance that three
games had been played?
\end{enumerate}

\subsubsection{Answer 1}
\label{sec-1-1-1}
The game can end in two turns if:
\begin{enumerate}
\item Teams draw the first time and then either one of them wins, the probability
of this happening is $0.1 * (0.6 + 0.3) = 0.09$.
\item Alternatively, if $A$ wins, either draw or repeated victory will do, 
the probability of this happening is $0.6 * (0.6 + 0.1) = 0.42$.
\item Finally, if $B$ wins, the probability of ending early is:
       $0.3 * (0.3 + 0.1) = 0.12$
\end{enumerate}

Summing up gives the total chances of the game ending in two rounds:
$0.09 + 0.42 + 0.12 = 0.63$.
\subsubsection{Answer 2}
\label{sec-1-1-2}
Since we already given that there was a third game, we will only concern
ourselves with three different possibilities this could've happen:
\begin{enumerate}
\item Two draws.
\item $A$ wins $B$ wins.
\item $B$ wins $A$ wins.
\end{enumerate}

It is easy to see (from commutativity of multiplication) that options (2)
and (3) are equally likely, thus the chance of $A$ being the first to
win is $(1 - 0.1 \times 0.1) / 2 = 0.495$.
\subsubsection{Answer 3}
\label{sec-1-1-3}
Similarly to the first answer, $A$ wins in two games if it either:
\begin{enumerate}
\item $A$ wins, $A$ wins, $p = 0.6 * 0.6 = 0.36$.
\item $A$ wins, draw, $p = 0.6 * 0.1 = 0.06$.
\item draw, $A$ wins, $p = 0.1 * 0.6 = 0.06$.
\end{enumerate}

All together: $0.48$.  The chance $A$ wins in three games is:
\begin{enumerate}
\item $A$ wins, $B$ wins $A$ wins, $p = 0.6 * 0.3 * 0.6 = 0.108$.
\item $B$ wins, $A$ wins, $A$ wins, $p = 0.3 * 0.6 * 0.6 = 0.108$.
\item draw, draw, $A$ wins, $p = 0.1 * 0.1 * 0.6 = 0.006$.
\end{enumerate}

All together: $0.222$. Which makes the total chances of $A$ to win $0.48 +
    0.222 = 0.702$, thus the fraction of the tournaments won by $A$ that lasted
for three rounds is $\frac{0.222}{0.702} = \num{0.31623932}$.

Below is the model testing the answers:

\lstset{language=Python,numbers=none}
\begin{lstlisting}
import random

def random_game():
    game = random.random()
    if game < 0.1: return 'draw'
    elif game < 0.4: return 'b'
    else: return 'a'

def finite_tournament():
    outcomes = []
    wins = ['drawa', 'drawb', 'aa', 'bb', 'adraw', 'bdraw']
    for g in range(3):
        outcomes.append(random_game())
        if ''.join(outcomes) in wins: break
    return outcomes

def game_ends_in_two(tries):
    two_rounds_games = 0
    for i in xrange(tries):
        game = finite_tournament()
        if len(game) == 2:
            two_rounds_games += 1
    return float(two_rounds_games) / tries

def a_wins_three_rounds_game(tries):
    games, a_wins = 0, 0
    for i in xrange(tries):
        game = finite_tournament()
        if len(game) > 2 and game[2] == 'draw':
            games += 1
            if game[0] == 'a': a_wins += 1
    return float(a_wins) / games

def at_least_three_games_played(times):
    good, games = 0, 0
    for i in xrange(times):
        game = finite_tournament()
        if game[-1] == 'a' or ''.join(game) == 'adraw':
            games += 1
            if len(game) > 2: good += 1
    return float(good) / games

print '''
+ Game ends in two turns: $\\num{%f}$
+ A wins first round: $\\num{%f}$
+ Three games were played: $\\num{%f}$
''' % (game_ends_in_two(100000),
       a_wins_three_rounds_game(100000),
       at_least_three_games_played(100000))
\end{lstlisting}

\begin{itemize}
\item Game ends in two turns: $\num{0.631370}$
\item A wins first round: $\num{0.486479}$
\item Three games were played: $\num{0.318114}$
\end{itemize}
\subsection{Problem 2}
\label{sec-1-2}
Some of the plates produced in a factory can be defective in two different
ways: with a chance of 0.15 there can be cracks in a plate and with a chance
of 0.25 the coloring of the plate may not be uniform.  The chance the
plate will be defective is 0.35.

\begin{enumerate}
\item One plate was found to be defective, what is the chance of this plate
to have cracks?
\item One plate was found to have cracks, what is the chance it will also
have uneven coating?
\item A plate was found to have no cracks, what is the chance of the plate
to be painted unevenly?
\end{enumerate}

\subsubsection{Answer 4}
\label{sec-1-2-1}
Total probability of having cracks is given to be 0.15, the probability of
being defective is 0.35, thus the chance of a plate having cracks, provided
it is defective is 0.15 in 0.35, i.e. $\num{0.428571428571}$.
\subsubsection{Answer 5}
\label{sec-1-2-2}
The chance of a plate having both cracks and uneven coating is one in three.
This is easy to see using the formula $P(A \cap B) = P(A) + P(B) - P(A \cup B)$.
Substituting gives: $P(A \cap B) = 0.15 + 0.25 - 0.35 = 0.05$.
\subsubsection{Answer 6}
\label{sec-1-2-3}
Of all plates 0.65 aren't defective, of the rest 0.1 have cracks, but are
painted properly (recall the result obtained in \ref{sec-1-2-2}.), thus the chance
of a plate to have been painted unevenly is $1 - 0.65 - 0.1 = 0.25$.
\subsection{Problem 3}
\label{sec-1-3}
Three coffee grinding machines produce all the coffee packed at a factory.
Machine $A$ grinds 0.55 of all the coffee, machine $B$ grinds 0.3 and machine
$C$ grinds the remaining 0.15 of coffee.  The coffee can be of fine or of a
coarce grind.  With a chance of 0.4, the machine $A$ produces fine grinds of
coffee.  The machine $B$ produces fine grinds with the 0.5 chance.
It is also known that the chance of producing fine grind of coffee overall
is 0.4.

\begin{enumerate}
\item A chosen pack of coffee was produced by machine $C$.  What is the chance
the coffee was ground finely?
\item A chosen pack of coffee was found to be of a fine grind.  What is the
chance it was produced by machine $B$?
\item Are events ``the coffee is finely ground'' and ``the coffee was ground
by the machine $A$'' are independent?
\end{enumerate}

\subsubsection{Answer 7}
\label{sec-1-3-1}
The chance of a pack of coffee to be ground finely, given it came from machine
$C$ is the total chance of coffee being ground finely sans the chance it
was ground finely and came from the machine $A$ or $B$.  Thus:
$x = \frac{0.4 - 0.55 \times 0.4 - 0.3 \times 0.5}{0.15} = 0.2$.
\subsubsection{Answer 8}
\label{sec-1-3-2}
The chance of a pack of coffee originating from machine $B$ is the chance
it was a finely ground coffe produced by machine $B$ divided by the total
chance it was finely ground: $x = \frac{0.3 \times 0.5}{0.4} = 0.375$.
\subsubsection{Answer 9}
\label{sec-1-3-3}
These events are not independent.  Independent events are such that their
intersection is an empty set, but there are clearly packs of coffee produced
by machine $B$, which are also finely ground \emph{(exactly half of them)}.
\subsection{Problem 4}
\label{sec-1-4}
Given a choice of three loaded coins, $A$ with a chance of tails being $\frac{1}{3}$,
$B$ with the chance of tails being $\frac{1}{2}$ and $C$ with the chance of tails
being $\frac{2}{3}$.  A random coin is selected.

\begin{enumerate}
\item What is the chance of tossing tails?
\item Same coin is tossed one more time, what is the chance it lends tails twice?
\item Given the coin landed tails twice, what is the chance the coin tossed
is the fair one?
\item Given the coin landed tails twice, what is the chance it will lend tails
again?
\end{enumerate}

\subsubsection{Answer 10}
\label{sec-1-4-1}
Since there is no preference towards any one of three coins, we will treat
the chance of choosing one as being equally likely.  Thus the chance of
tossing tails is simply the average of the three:
$x = \Big(\frac{1}{3} + \frac{1}{2} + \frac{2}{3}\Big) \times \frac{1}{3} = \frac{1}{2}$.
\subsubsection{Answer 11}
\label{sec-1-4-2}
There is an equal chance to select all coins, therefore we will need to
divide the total in three.  For each coin the chance of throwing tails
subsequently is given by the probability of throwing tails squared, this
gives: $(\frac{2}{3})^2 \times (\frac{1}{2})^2 \times (\frac{1}{2})^2)
    \times \frac{1}{3} = \frac{29}{108}$.
\subsubsection{Answer 12}
\label{sec-1-4-3}
We can calculate how each one of the coins contributes towards the total
chance of tossing tails (obtained in the previous answer).  This is given by
$P(A) = \frac{1}{9}$, $P(C) = \frac{4}{9}$ and $P(B) = \frac{1}{4}$.  The
probability of choosing the fair coin given it landed tails twice is the
$\frac{P(B)}{P(A) + P(B) + P(C)} = \frac{1}{4} \times \frac{36}{29} = \frac{9}{29}$.
\subsubsection{Answer 13}
\label{sec-1-4-4}
The fact that the coin landed tails twice shifts the probabilities of that
coin towards being the $C$ coin.  We already calculated the total probability
of a coin landing tails after two tosses (in \ref{sec-1-4-2}), now we need to find
how much every coin contributes to that probability and multiply that with
the probability of each coin landing on tails.  To get single contributions
we will multiply each contribution with the total's inverse: $\frac{108}{29}$.
$\frac{2 * 108}{3 * 29} + \frac{108}{2 * 29} * \frac{108}{3 * 29} = \frac{162}{29}$.

Below is the code that models the coin tosses:

\lstset{language=Python,numbers=none}
\begin{lstlisting}
import random

def select_coin():
    tails = 0
    chance = random.random() * 3
    if chance <= 1: tails = 1.0 / 3
    elif chance <= 2: tails = 0.5
    else: tails = 2.0 / 3
    return tails

def chance_of_tails(times):
    return sum(select_coin() for x in xrange(times)) / times

def chance_two_tails(times):
    return sum(select_coin() ** 2
               for x in xrange(times)) / times

def chance_of_fair_two_tails(times):
    fair, total = 0.0, 0.0
    for x in xrange(times):
        coin = select_coin() ** 2
        if coin == 0.25: fair += coin
        total += coin
    return fair / total

def chance_third_tails(times):
    two_thirds, half, third, total = 0.0, 0.0, 0.0, 0.0
    for x in xrange(times):
        coin = select_coin() ** 2
        if coin == 0.25: half += coin
        elif coin < 0.25: third += coin
        else: two_thirds += coin
        total += coin
    two_thirds /= total
    half /= total
    third /= total
    return two_thirds * 2 / 3 + half / 2 + third / 3

print '''
+ Probability of tossing tails: $\\num{%s}$
+ Probability of tossing two tails: $\\num{%s}$
+ Chance of fair coin given two tails: $\\num{%s}$
+ Chance third toss will be tails: $\\num{%s}$
''' % tuple(f(100000)
            for f in [chance_of_tails,
                      chance_two_tails,
                      chance_of_fair_two_tails,
                      chance_third_tails])
\end{lstlisting}

\begin{itemize}
\item Probability of tossing tails: $\num{0.499205}$
\item Probability of tossing two tails: $\num{0.268951944444}$
\item Chance of fair coin given two tails: $\num{0.31315176421}$
\item Chance third toss will be tails: $\num{0.568445329413}$
\end{itemize}
% Emacs 25.0.50.1 (Org mode 8.2.2)
\end{document}